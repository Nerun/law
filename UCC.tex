% ===============================================================
% PREÂMBULO =====================================================
% ===============================================================
\documentclass[a4paper, 12pt]{article}
\usepackage[top=2cm, bottom=2cm, left=2cm, right=2cm]{geometry}
\usepackage{setspace}
\singlespacing
\usepackage{indentfirst}

\usepackage[utf8]{inputenc}
\usepackage[T1]{fontenc}
\usepackage[brazilian]{babel}

\setlocalecaption{brazilian}{contents}{\begin{center}Sumário\end{center}}
\setcounter{secnumdepth}{3}

\usepackage{enumitem}
\setlist[enumerate]{itemsep=1pt, parsep=1pt, topsep=1pt}
\setlist[itemize]{itemsep=1pt, parsep=1pt, topsep=1pt}

\usepackage{amsmath, amsfonts, amssymb}
\usepackage{xcolor,soul}
\usepackage{lipsum}

\newcommand{\autor}{Daniel Dias Rodrigues}
\newcommand{\titulo}{Código Comercial Uniforme}

% CABEÇALHO
\title{\textbf{\titulo}}
\author{\textit{tradução de}\\\autor}
\date{13 de Janeiro de 2021}

\usepackage[pdftex,
            pdfauthor={\autor},
            pdftitle={\titulo},
            pdfsubject={Tradução de legislação americana},
            pdfkeywords={Lei, Direito, Estados Unidos},
            pdfproducer={Texmaker 5.0.3 (pdfTeX 1.40.22)},
            pdfcreator={LaTeX com hyperref}]{hyperref}
\hypersetup{
    colorlinks=true,
    linkcolor=black,
    filecolor=magenta,
    urlcolor=blue,
}

% ===============================================================
% CORPO DO DOCUMENTO ============================================
% ===============================================================
\begin{document}

\maketitle

Esta é uma tradução do \href{https://www.law.cornell.edu/ucc}{\underline{Uniform Commercial Code}} conforme publicado pelo \textit{Legal Information Institute} da Universidade Cornell.

\begin{center}
\rule{7cm}{0.4pt}
\end{center}

\textit{Copyright \textcopyright 1978, 1987, 1988, 1990, 1991, 1992, 1994, 1995, 1998, 2001, 2004, 2010, 2011, 2012 pelo ``American Law Institute" e pela ``National Conference of Commissioners on Uniform State Laws"; reproduzido, publicado e distribuído com a permissão do Conselho Editorial Permanente do Código Comercial Uniforme para os fins limitados de estudo, ensino e pesquisa acadêmica.}

\vspace{5mm}

\textit{O texto desta tradução é copyright \textcopyright 2021 de Daniel Dias Rodrigues. Alguns direitos reservados. Os direitos autorais sobre traduções são protegidos pela Convenção de Berna art. 2, alínea 3 (Decreto nº 75.699/75) e pela Lei de Direitos Autorais (``LDA") art. 7º caput e inciso XI (Lei Federal nº 9.610/98). Esta tradução é distribuída sob os termos da licença \href{https://creativecommons.org/licenses/by/4.0/deed.pt_BR}{\underline{\smash{Creative Commons BY 4.0}}}. Por essa licença tudo é permitido, inclusive alterar a tradução e lucrar com ela sem ter que pagar royalties, desde que você cite o nome do tradutor. Também pela LDA art. 53, inciso II, além do título original, você também é obrigado a citar o nome do tradutor. O tradutor é titular de direitos autorais (LDA art. 14). Isso é um direito moral do autor (LDA art. 24, II). Os direitos morais do autor são inalienáveis e irrenunciáveis (LDA art. 27).}

\vspace{5mm}

Para erros de tradução: \href{mailto:danieldiasr@gmail.com}{\underline{\smash{danieldiasr@gmail.com}}}.

\begin{center}
\rule{7cm}{0.4pt}
\end{center}

Nossa coleção tem como objetivo disponibilizar o C.C.U. na versão mais amplamente adotada pelos estados. \textbf{Isso significa que nem sempre usaremos a revisão mais recente se essa revisão não tiver alcançado ampla adoção entre as legislaturas americanas.}

\vspace{5mm}

Devido a restrições de licença, esta versão do C.C.U. não inclui os comentários oficiais.

\pagebreak

\tableofcontents

\pagebreak

\section{ARTIGO 2 - VENDAS (2002)}

\subsection{PARTE 3 - OBRIGAÇÃO GERAL E FORMAÇÃO DO CONTRATO}

\subsubsection{§ 2-315. Garantia implícita: adequação a uma finalidade específica.}

Quando o vendedor, no momento da contratação, tiver conhecimento de qualquer propósito em particular para o qual o produto é necessário e de que o comprador está contando com a habilidade ou julgamento do vendedor para selecionar ou fornecer o produto adequado, então haverá, a não ser que seja excluída ou modificada na próxima seção, uma garantia implícita de que os produtos serão adequados para esse fim.

\subsubsection{§ 2-316. Exclusão ou modificação de garantias.}

\begin{enumerate}[label=(\arabic*)]
	\item Palavras ou conduta relevantes para a criação de uma garantia expressa e palavras ou conduta tendentes a negar ou limitar a garantia, devem ser interpretadas sempre que possível como consistentes umas com as outras; mas, sujeito às disposições deste Artigo sobre liberdade condicional ou evidência extrínseca (Seção 2-202), a negação ou limitação não terá efeito na medida em que tal interpretação não seja razoável.
	\item Sujeito à subseção (3), para excluir ou modificar a garantia implícita de comerciabilidade ou qualquer parte dela, o texto deve mencionar a comerciabilidade de forma conspícua, e para excluir ou modificar qualquer garantia implícita de adequação, a exclusão deve ser por escrito e conspícua. A linguagem para excluir todas as garantias implícitas de adequação será suficiente se ela declarar, por exemplo, que ``Não há garantias que se estendam para além daquelas aqui apresentadas".
	\item Não obstante a subseção (2):
	\begin{enumerate}
		\item a menos que as circunstâncias indiquem o contrário, todas as garantias implícitas são excluídas por expressões como ``no estado em que se encontra", ``com todas as falhas" ou outra redação que, no entendimento comum, chama a atenção do comprador para a exclusão de garantias e deixa claro que não há nenhuma garantia implícita; e
		\item quando o comprador, antes de celebrar o contrato, tiver examinado o produto, a amostra ou o modelo tão completamente quanto ele desejava, ou tiver se recusado a examinar o produto, não haverá garantia implícita no que diz respeito aos defeitos que um exame deveria, nessas circunstâncias, ter revelado para ele; e
		\item uma garantia implícita também pode ser excluída ou modificada pelo curso da negociação, pelo curso do desempenho ou pelos usos e costumes do comércio.
	\end{enumerate}
	\item As indenizações por violação de garantia podem ser limitadas de acordo com as disposições deste Artigo sobre liquidação ou limitação de danos e sobre modificação contratual das indenizações (Seções 2-718 e 2-719).
\end{enumerate}

\end{document}
