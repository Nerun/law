% ===============================================================
% PREÂMBULO =====================================================
% ===============================================================
\documentclass[a5paper,10pt]{book}
\usepackage[margin=1.5cm]{geometry}
\usepackage{setspace}
\singlespacing
\usepackage{indentfirst}

\usepackage[utf8]{inputenc}
\usepackage[T1]{fontenc}
\usepackage[brazilian]{babel}

\setlocalecaption{brazilian}{contents}{\centering Sumário}
\setcounter{secnumdepth}{-2}

\usepackage{enumitem}
\setlist[enumerate]{itemsep=1pt, parsep=1pt, topsep=1pt}
\setlist[itemize]{itemsep=1pt, parsep=1pt, topsep=1pt}

\usepackage{amsmath, amsfonts, amssymb}
\usepackage{xcolor,soul}
\usepackage{lipsum}

\newcommand{\sublinhado}[1]{\underline{\smash{#1}}}

\newcommand{\autor}{Daniel Dias Rodrigues}
\newcommand{\titulo}{Código dos Estados Unidos}

% CABEÇALHO
\title{\textbf{\titulo}}
\author{\textit{tradução de}\\\autor}
\date{18 de Abril de 2022}

\usepackage[pdftex,
            pdfauthor={\autor},
            pdftitle={\titulo},
            pdfsubject={Tradução de legislação americana},
            pdfkeywords={Lei, Direito, Estados Unidos},
            pdfproducer={Texmaker 5.0.3 (pdfTeX 1.40.22)},
            pdfcreator={LaTeX com hyperref}]{hyperref}
\hypersetup{
    colorlinks=true,
    linkcolor=black,
    filecolor=magenta,
    urlcolor=blue,
}

% ===============================================================
% CORPO DO DOCUMENTO ============================================
% ===============================================================
\begin{document}

\maketitle

Esta é uma tradução do \href{https://uscode.house.gov}{\sublinhado{United States Code}} conforme publicado pelo \textit{Office of the Law Revision Counsel of the United States House of Representatives} (Escritório do Conselho de Revisão da Lei da Câmara dos Deputados dos Estados Unidos).

\textit{O texto desta tradução é copyright \textcopyright 2022 de Daniel Dias Rodrigues. Alguns direitos reservados. Os direitos autorais sobre traduções são protegidos pela Convenção de Berna art. 2, alínea 3 (Decreto nº 75.699/75) e pela Lei de Direitos Autorais (``LDA'') art. 7º caput e inciso XI (Lei Federal nº 9.610/98). Esta tradução é distribuída sob os termos da licença \href{https://creativecommons.org/licenses/by/4.0/deed.pt_BR}{\underline{\smash{Creative Commons BY 4.0}}}. Por essa licença tudo é permitido, inclusive alterar a tradução e lucrar com ela sem ter que pagar royalties, desde que você cite o nome do tradutor. Também pela LDA art. 53, inciso II, além do título original, você também é obrigado a citar o nome do tradutor. O tradutor é titular de direitos autorais (LDA art. 14). Isso é um direito moral do autor (LDA art. 24, II). Os direitos morais do autor são inalienáveis e irrenunciáveis (LDA art. 27).}

Para erros de tradução: \href{mailto:danieldiasr@gmail.com}{\underline{\smash{danieldiasr@gmail.com}}}.

\tableofcontents

\part[TÍT. 7 -- DIREITOS AUTORAIS (``COPYRIGHTS'')]{Título 7\\ DIREITOS AUTORAIS (``COPYRIGHTS'')}

\chapter[Cap. 1 -- ASSUNTO E ESCOPO DOS DIREITOS AUTORAIS]{Capítulo 1\\ ASSUNTO E ESCOPO DOS DIREITOS AUTORAIS}

\section{§ 102 - Assunto dos direitos autorais: em geral}

\begin{enumerate}[label=(\alph*)]
	\item A proteção de direitos autorais subsiste, de acordo com este título, em obras originárias fixadas em qualquer meio de expressão tangível, agora conhecido ou desenvolvido posteriormente, a partir do qual possam ser percebidas, reproduzidas ou de outra forma comunicadas, diretamente ou com o auxílio de uma máquina ou dispositivo. Obras originárias incluem as seguintes categorias:
	\begin{enumerate}[label=(\arabic*)]
		\item obras literárias;
		\item obras musicais, incluindo quaisquer palavras acompanhantes;
		\item obras dramáticas, incluindo qualquer música acompanhante;
		\item pantomimas e trabalhos coreográficos;
		\item obras pictóricas, gráficas e escultóricas;
		\item filmes e outras obras audiovisuais;
		\item gravações de som; e
		\item obras arquitetônicas.
	\end{enumerate}
	\item Em nenhum caso a proteção de direitos autorais para uma obra originária se estende a qualquer ideia, procedimento, processo, sistema, método de operação, conceito, princípio ou descoberta, independentemente da forma com que são descritos, explicados, ilustrados, ou incorporados em tal trabalho.
\end{enumerate}

\end{document}
